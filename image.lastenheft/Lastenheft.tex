\documentclass[parskip=full]{scrartcl}
\usepackage[utf8]{inputenc} % use utf8 file encoding for TeX sources
\usepackage[T1]{fontenc}    % avoid garbled Unicode text in pdf
\usepackage[german]{babel}  % german hyphenation, quotes, etc
\usepackage{hyperref}       % detailed hyperlink/pdf configuration
\usepackage{graphicx}       % provides commands for including figures
\usepackage{csquotes}       % provides \enquote{} macro for "quotes"
\usepackage[nonumberlist]{glossaries}     % provides glossary commands
\usepackage{enumitem}
\usepackage{tikz}
\usepackage{tikz-uml}
\usepackage{tikzsymbols}

\makenoidxglossaries
%
% % Glossareinträge
%
\newglossaryentry{Applikation}
{
	name=Applikation,
	plural=Applikationen,
	description={oder auch Anwendungssoftware werden Computerprogramme bezeichnet, die genutzt werden, um eine nützliche oder gewünschte nicht systemtechnische Funktionalität zu bearbeiten oder zu unterstützen. (Beispiele für Anwendungsgebiete sind: \enquote{Bildbearbeitung, E-Mail-Programme, Webbrowser, Textverarbeitung, Tabellenkalkulation oder Computerspiele})}
}

\newglossaryentry{iMage}
{
	name=iMage,
	plural=-,
	description={ist eine Anwendungssoftware der Firma Pear Corp und dient dem Anwenden von Kunstfiltern auf bestimmte Bilder}
}

\newglossaryentry{Komprimierung}
{
	name=Komprimierung,
	plural=Komprimierungen,
	description={ist ein Vorgang um Dateigrößen zu verringern}
}

\newglossaryentry{Domain}
{
    name=Domain,
    description={ist vereinfacht gesagt ein Name für einen gewissen Netzwerkbereich (z.B. \enquote{führt google.com zur Google Suchmaschine})}
}

\title{SWT1: Lastenheftvorlage}
\author{Nils Pukropp, 2301588}

\begin{document}

\maketitle

%
% % Hier beginnt die Gliederung des Lastenhefts
%
\section{Zielbestimmung}
Die Firma Pear Corp soll durch das Produkt in die Lage versetzt werden, Bilder automatisiert aus dem Internet herunterzuladen.

\section{Produkteinsatz}
Das Produkt dient der Firma Pear Corp zum Beschaffen von frei verfügbaren Bildern übers Internet.

Zielgruppe: Nutzer der \gls{Applikation} \gls{iMage}

Plattform: Arch, React, Nintendo Switch, Playstation 4, xBox One X

\section{Funktionale Anforderungen}
\begin{itemize}[nosep]
\item[FA10] Das Suchen von Bildern mit Kriterien(Anzahl, Nutzungsrechte, Dateiformat, Dateigröße)
\item[FA20] Das Suchen von Bildern mit integrierter \gls{Komprimierung}
\item[FA30] Das Suchen von Bildern in angegebenen (Sub-)\glspl{Domain}
\item[FA40] Das Anzeigen der über die Sucher geladenen Bilder mit Kriterien(mittlerer Farbwert, Name, Herkunft)
\item[FA50] Die Bilder können lokal oder auf den Pear Crop Zentralservern gespeichert werden.
\end{itemize}

\section{Produktdaten}
\begin{itemize}[nosep]
\item[PD10] Es sind relevante Daten über die Nutzer zu speichern.
\item[PD20] Es sind relevante Daten über die geladenen Bilder zu speichern
\item[PD30] Es sind Bilder lokal oder über den Zentralserver zu speichern
\end{itemize}

\section{Nichtfunktionale Anforderungen}
\begin{itemize}[nosep]
\item[NF10] Die Funktion /FA10/ soll für 500 Bilder maximal 10 Minuten benötigen und selbständig nach einer Suchdauer von einer Stunde abbrechen.
\item[NF20] Die Funktion /FA20/ soll für 500 Bilder maximal 20 Minuten benötigen und selbständig nach einer Suchdauer von zwei Stunden abbrechen.
\item[NF30] Der Zugriff auf den Zentralserver von Pear Corp soll von mindestens einhundert (100) Nutzern gleich-
zeitig erfolgen können und die Dauer des Hochladens der Bilder maximal linear mit der Anzahl der Bilder
wachsen.
\end{itemize}

\section{Systemmodelle}

\subsection{Szenarien}

\subsection{Anwendungsfälle}
\subsubsection{Bildersuchen/-anzeigen}
% I have no idea what I'm doing, send help pls :( 
\begin{tikzpicture}
    \fill[blue,opacity=0.3] (2,0) rectangle ++(6,9);
    \umlsimpleclass[x=5,y=7, width=20ex]{Bildersuche}
    \umlsimpleclass[x=5,y=1, width=20ex]{Bilder anzeigen}
    \umlsimpleclass[x=5,y=4]{Geladene Bilder}
    \path (-0.3,4.4) node[above,inner sep=0pt] {\Strichmaxerl[7]} node[below] {Nutzer};
    \path (11,4.4) node[above,inner sep=0pt] {\Ofen[5]} node[below] {Server};
    \coordinate (user) at (0,5);
    \coordinate (server) at (10,4);
    \node[draw] at (5,8.5) {Applikation};
    \draw (user) edge[->, thick,shorten >=2pt,shorten <=2pt] node[sloped, anchor=center, above]{Bilder suchen} (3.2,7);
    \draw (Bildersuche) edge[->, thick,shorten >=2pt,shorten <=2pt] node[right] {Bilder laden}(Geladene Bilder);
    \draw (Geladene Bilder) edge[->, thick, shorten >=2pt, shorten <=2pt] node[right] {Bilder laden} (Bilder anzeigen);
    \draw (Bilder anzeigen) edge[bend right,->, thick, shorten >=2pt, shorten <=2pt] node[sloped, anchor=center,below] {Bilder auf Server speichern} (server);
    \draw (Bilder anzeigen) edge[bend left,->, thick, shorten >=2pt, shorten <=2pt] node[sloped, anchor=center,below] {Bilder lokal speichern} (0,3.9);
    \draw (0,3.9) edge[bend left,->, thick, shorten >=2pt, shorten <=2pt] node[sloped, anchor=center,above] {Bilder anzeigen} (3.35,1.35);
\end{tikzpicture}

Akteure: Nutzer, Server.

Anwendungsfälle: Bildersuche, Bilder anzeigen, Bilder speichern.

Textuelle Beschreibung: (folgt)



%
% % Automatisch generiertes Glossar (Latex zwei mal ausführen um Glossar anzuzeigen)
%
%\glsaddall % das sorgt dafür, dass alles Glossareinträge gedruckt werden, nicht nur die verwendeten. Das sollte nicht nötig sein!
\printnoidxglossaries
Siehe \url{https://en.wikibooks.org/wiki/LaTeX/Glossary}.




\end{document}
