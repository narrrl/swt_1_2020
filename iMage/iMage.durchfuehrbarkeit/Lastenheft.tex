\documentclass[parskip=full]{scrartcl}
\usepackage[utf8]{inputenc} % use utf8 file encoding for TeX sources
\usepackage[T1]{fontenc}    % avoid garbled Unicode text in pdf
\usepackage[german]{babel}  % german hyphenation, quotes, etc
\usepackage{hyperref}       % detailed hyperlink/pdf configuration
\usepackage{graphicx}       % provides commands for including figures
\usepackage{csquotes}       % provides \enquote{} macro for "quotes"
\usepackage[nonumberlist]{glossaries}     % provides glossary commands
\usepackage{enumitem}

\title{SWT1: Durchführbarkeitsuntersuchung}
\author{Nils Pukropp, 2301588}

\begin{document}

\maketitle

%
% % Hinweise - sollen nicht im endgültigen Dokument erscheinen, daher vor der Abgabe löschen!
%
\section{Fachliche Durchführbarkeit}
\begin{flushleft}
	Die Applikation setzt keine große fachliche Kompetenz und sollte mit unserem Wissen umgesetzt werden können.
	Durch Java sind wir auch an keine Platform gebunden und müssen uns keine Sorgen machen über Entwicklungs- und Zielmaschinen.
\end{flushleft}

\section{Alternative Lösungsvorschläge}
\begin{flushleft}
	Es steht eine große Auswahl an automatisierten Bilder-Downloadern im Internet zur Verfügung die ohne große Kosten erworben werden könnten, weswegen sich die Entwicklung eventuell nicht lohnt.
	Die kosten unseres Personals würden deutlich höher ausfallen, aber eine gekaufte Software müsste natürlich auch wieder von uns in unser bestehendes System integriert werden, weswegen die beiden Optionen quasi gleichwertig sind.
\end{flushleft}

\section{Personelle Durchführbarkeit}
\begin{flushleft}
	Wie bereits erwähnt sollte die Applikation keine sonderlich hohen Kompetenzen vorraussetzen und deswegen einfach von unserem Personal umgesetzt werden. Es eignet sich womöglich bereits eine Einzelperson zum implementieren.

\end{flushleft}

\section{Risiken}
\begin{flushleft}
	Das Projekt hat eher wenig Risiken, denn selbst wenn das Projekt nicht funktionieren sollte oder spontan pausiert oder abgebrochen wird, sollte durch die einfache Implementierung und den geringen personellen Aufwand kein sonderlich großer Schaden entstehen.
	Auch wenn das Projekt am Ende nicht gebraucht wird, könnte man es durch seine universelle Verwendbarkeit einfach verkaufen.
\end{flushleft}

\section{Ökonomischen Durchführbarkeit}
\begin{flushleft}
	Durch die einfache Implementierung und geringen personellen Aufwand sollte das Projekt keinen größeren Zeitraum beanspruchen und keinem großen Aufwand entsprechen. Damit fällt auch die Finanzierung eher einfach aus.
\end{flushleft}

\section{Rechtliche Gesichtspunkte}
\begin{flushleft}
	Der rechtliche Aspekt wird wahrscheinlich das größte Problem am Projekt, da hinter Bildern häufig Lizenzen stecken die nicht einfach umgangen werden sollten. 
\end{flushleft}


\end{document}
